\section{はじめに} \label{section:introduction}
コンピュータが搭載しているメモリの使用量は増加しており,それに伴ってインメモリ分散処理フ
レームワークやインメモリデータベースなどの膨大なメモリを利用するアプリケーションが登場して
きている.インメモリ分散処理フレームワーク\cite{zaharia2010spark}とは,分散処理フレームワーク\cite{dean2008mapreduce}の一種で,処理
をできる限りメモリ内で行うことで,従来のものよりも処理速度などを向上させたシステムである.分
散処理フレームワークでは,複数のコンピュータをネットワークでつなぎ,処理を分散して行うこと
で,1 つのサーバではさばききれないような非常に大きなデータを処理することができ,サーバを用意
するコストが抑えられることや,リソースの拡張が容易であるといったメリットが存在する.中でもイ
ンメモリ分散処理フレームワークは,膨大なメモリを利用して,処理をメモリ内で完結させることで,
従来のメリットに加えて,ディスクI/O やネットワークI/O にかかるボトルネックを減らすことがで
きる.インメモリ分散処理フレームワークにおいて膨大なメモリが必要な理由として,通常の分散処理
フレームワークとの処理方式の違いが挙げられる.分散処理フレームワークでは,データの処理を行う
たびにディスクへの書き出しを行う.対して,インメモリ分散処理フレームワークでは,データの処理
が終わった後,ディスクへの書き出しを行わず,そのままメモリ内に保持し続けることで処理の高速化
を実現している.このデータを保持するために膨大なメモリが必要となる.実際に多くのメモリを搭載
するマシンの例として,Amazon EC2 が提供するHigh Memory インスタンス\cite{amazon-ec2-high-memory}では,最大24TB
ものメモリを搭載しており,この巨大なサイズのメモリをインメモリ分散処理フレームワークなどのア
プリケーションが利用することになる.

インメモリ分散処理フレームワークなどの膨大なメモリを利用するアプリケーションでは,アドレス
変換が大きなボトルネックになってしまう\cite{basu2013efficient},\cite{gandhi2014efficient}.
OS にはページングという機構があり,ページという単位で仮想アドレスと物理アドレスの変換を行う.
高速化のために一度変換したページはTLB という高速なハードウェアに保存することでアドレス変換を高速に
行うことが可能となるが,TLBはサイズがあまり大きくない.そのため,膨大なメモリに対してはTLBに
キャッシュしておける量が少なくなり,TLBを使ってアドレス変換できる範囲が狭く,多くのアドレス変換は
TLBを使わない遅いものとなってしまう.

このような問題点を解決するために,ヒュージページという機構が存在する.ヒュージページでは,
ページのサイズを,通常よりも大きいサイズで使用することで一度に変換できるアドレスの範囲を大きくする
ことができる.これによってTLB に保存できるアドレスの範囲が増え,多くのページで高速なアクセスが可能となり,
他にもページテーブルサイズの削減やページウォーク時間の短縮といった効果が得られる.

インメモリ分散処理フレームワークでは膨大なメモリを利用することから,ヒュージページの恩恵を
期待できる.しかし,インメモリ分散処理フレームワークにヒュージページを適用する研究は存在しおらず,
効率的なヒュージページの使用法はわかっていない.
また,インメモリ分散処理フレームワークでヒュージページを適用するのはアプリケーション単位でしかできないという
問題点もある.
インメモリ分散処理フレームワークでヒュージページを使うにはTHPという動的なヒュージページ割り当てを行う
機構を利用する.実行時にJavaオプションでJVMごとにヒュージページの使用の有無を指定することで扱うことができる\cite{java-thp}.
しかし,THPにはメモリ肥大化やメモリコンパクションのためのCPU占有などのデメリットが存在し,インメモリ分散処理フレームワークの
特性を考慮せずアプリケーション全体に闇雲に割り当てることが正しいとは限らない.

事前研究としてインメモリ分散処理フレームワークのヒュージページの効果を調査を行った.
調査の結果,ヒュージページはインメモリ分散処理フレームワークで効果的に働き実行時間を短縮させた.
しかし,アプリケーションによっては,繰り返し利用されるような一部のメモリに対してのみヒュージページ割り当てを
行った場合でも実行時間の削減率には大きな差が出ないこともわかった.

本研究ではインメモリ分散処理フレームワーク上でヒュージページを効率的に利用できる手法を提案する.
インメモリ分散処理フレームワークには長期間保存され繰り返し利用されるデータや一時的に使用される中間データなどの
特徴があり中間データではヒュージページの効果が薄く,繰り返し利用されるデータに対してヒュージページを
割り当てるべきだと考えられる.
提案方式では,そのような,繰り返し利用されるデータにヒュージページの割り当てを行い,それ以外では通常のページ割り当てを
行うことで無駄なヒュージページ割り当てを避ける.
これにより,メモリ肥大化やコンパクションなどのデメリットを抑えつつ,ヒュージページによる性能向上の恩恵を受けることを狙う.

本論文の貢献は以下の通りである.
\begin{itemize}
  \item ヒュージページを有効的に活用するインメモリ分散処理フレームワークを提案した.長期間生存して再利用されやすいデータに対してヒュージページ割り当てを行う
  \item 提案を実現するためにJVMとSparkの改良を行った.JVM側ではヒュージページでデータを生成する命令を作成しSpark側でヒュージページに適したオブジェクトをこの命令を用いて作成する
  \item マイクロベンチマークを作成して実験を行った.実験の結果,一部のメモリに対してヒュージページ割り当てを行うことができ,全体割り当てよりも実行時間を削減した
\end{itemize}
