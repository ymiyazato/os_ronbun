\section{はじめに} \label{section:introduction}
Spark に代表されるインメモリ分散処理フレームワーク~\cite{zaharia2010spark}は,マシンに搭載されている膨大なメモリを活用するアプリケーションを構築するフレームワークである.インメモリ分散処理フレームワークは分散処理フレームワーク\cite{dean2008mapreduce}の一種で,各マシン内での処理を可能な限りメモリ内で行うことで,巨大なメモリを要するアプリケーションの実現やストレージ I/O を抑えながらの処理速度向上が期待できる.たとえば,巨大なグラフを解析したり,大量のデータを必要とする機械学習などのアプリケーションの実現に利用される.インメモリ分散処理フレームワークにおいて膨大なメモリが必要な理由として,通常の分散処理フレームワークとの処理方式の違いが挙げられる.分散処理フレームワークでは,データの処理を行うたびにストレージへの書き出しを行う.対して,インメモリ分散処理フレームワークでは,データの処理が終わった後,ディスクへの書き出しを行わず,そのままメモリ内に保持し続けることで処理の高速化を実現している.このデータを保持するために膨大なメモリが必要となる.実際に多くのメモリを搭載するマシンの例として,Amazon EC2 が提供するHigh Memory インスタンス\cite{amazon-ec2-high-memory}では,最大24TBものメモリを搭載しており,この巨大なサイズのメモリをインメモリ分散処理フレームワークなどのアプリケーションが利用することになる.

インメモリ分散処理フレームワークなどの膨大なメモリを利用するアプリケーションでは,アドレス変換が大きなボトルネックとなることが知られている~\cite{basu2013efficient, gandhi2014efficient}.メモリ管理にはページングが広く利用されており,ページという単位で仮想アドレスと物理アドレスの変換を行う.高速化のために一度変換したページは TLB という高速なハードウェアに保存することでアドレス変換を高速に行うことが可能となるが,TLBはサイズがあまり大きくない.そのため,膨大なメモリに対してはTLBにキャッシュしておける量が少なくなり,TLBを使ってアドレス変換できる範囲が狭く,多くのアドレス変換は TLB を使わない遅いものとなってしまう.

このような問題点を解決するために,ヒュージページという機構が存在する.ヒュージページでは,ページのサイズを通常よりも大きいサイズで使用することで一度に変換できるアドレスの範囲を大きくすることができる.これによってTLB に保存できるアドレスの範囲が増え,多くのページで高速なアクセスが可能となり,他にもページテーブルサイズの削減やページウォーク時間の短縮といった効果が得られる.

インメモリ分散処理フレームワークでは膨大なメモリを利用することから,ヒュージページの恩恵を期待できる.しかし,インメモリ分散処理フレームワークにヒュージページを適用する研究は存在しおらず,効率的なヒュージページの使用法はわかっていない.また,インメモリ分散処理フレームワークでヒュージページを適用するのはアプリケーション単位でしかできないという問題点もある.インメモリ分散処理フレームワークでヒュージページを使うにはTHPという動的なヒュージページ割り当てを行う機構を利用する.実行時にJavaオプションでJVMごとにヒュージページの使用の有無を指定することで扱うことができる\cite{java-thp}.しかし,THPにはメモリ肥大化やメモリコンパクションのためのCPU占有などのデメリットが存在し,インメモリ分散処理フレームワークの特性を考慮せずアプリケーション全体に闇雲に割り当てることが正しいとは限らない.

本研究では,インメモリ分散処理フレームワークのヒュージページの効果の定量的調査を行い,インメモリ分散処理フレームワーク上でヒュージページを効率的に利用できる手法を提案する.Linux の Transparent Hugepage (THP) および Spark XXXX を用いた調査においては,ヒュージページの割当部分を工夫することでメモリ使用量の増加を抑えながらインメモリ分散処理フレームワークの実行時間を最大で XXXX\% 短縮させた.一方でアプリケーションによっては,繰り返し利用されるような一部のメモリに対してのみヒュージページ割り当てを行った場合でも実行時間の削減率には大きな差が出ないこともわかった.本調査の結果を受け,ヒュージページの割り当てが効果的な部分のみにヒュージページを適用し,それ以外では通常のページ割り当てを行うことで無駄なヒュージページ割り当てを避ける機構を提案する.提案機構は言語ランタイム上で稼働し,メモリ肥大化やコンパクションなどのデメリットを抑えつつ,ヒュージページによる性能向上の恩恵を受けることを狙う.具体的には,インメモリ分散処理フレームワークには長期間保存され繰り返し利用されるデータや一時的に使用される中間データなどの特徴があり中間データではヒュージページの効果が薄く,繰り返し利用されるデータに対してヒュージページを割り当てる.

本論文の貢献は以下の通りである.
\begin{itemize}
  \item Real-world のインメモリ分散処理フレームワークにおいて,ヒュージページの効果を定量的に示した.
  \item ヒュージページを有効的に活用するインメモリ分散処理フレームワークを提案した.長期間生存して再利用されやすいデータに対してヒュージページ割り当てを行う.
  \item 提案を OpenJDK X.X.X とSpark X.X.X 上に実装した.JVM側ではヒュージページでデータを生成する命令を作成しSpark側でヒュージページに適したオブジェクトをこの命令を用いて作成する.
  \item マイクロベンチマークを作成して実験を行った.実験の結果,一部のメモリに対してヒュージページ割り当てを行うことができ,全体割り当てよりも実行時間を削減した.
\end{itemize}
