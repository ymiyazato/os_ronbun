\section{おわりに} \label{section:conclusion}
本研究ではヒュージページを活用するインメモリ分散処理フレームワークを提案した.
ヒュージページはインメモリ分散処理フレームワークのような膨大なメモリを利用するアプリケーションと
相性がよくパフォーマンスの向上が期待できる.
しかし,現在インメモリ分散処理フレームワークでヒュージページを使うにはアプリケーション単位でしか
選択できず,データ単位で割り当てることができない.
THPにはメモリ肥大化やメモリコンパクションのためのCPU占有などのデメリットが存在するため
闇雲にアプリケーション全体に割り当てるのが正しいとは限らず,Sparkのデータ特性に合わせた
割り当てが必要となってくる.
提案方式ではデータ単位でのヒュージページ割り当てを実装し,長期間生存して再利用されやすい
データに対してのみヒュージページ割り当てを行うことで,ヒュージページの恩恵を受けつつ,デメリットを抑えることを狙う.
提案方式をApache SparkとOpenJDKに実装しマイクロベンチマークで実験を行った.
実験の結果,実装したシステムが正常に動作し,一部のメモリにのみヒュージページを割り当てることに
成功した.そして,提案方式が全体割り当てよりも実行時間を削減できたことを確認した.
今後として,マイクロベンチマーク以外のPagerank\cite{page1999pagerank}やkmeansといったReal-timeなワークロードの
実験を行う.また,提案方式ではユーザが自身のコード内で生成したオブジェクトをヒュージページに
するためにはコードをユーザが変更する必要があった.既存のコードの再利用などのために,
StrageMmeoryにキャッシュされるオブジェクトについては全てを自動でヒュージページにするように
改良を行う必要があると考えられる.


