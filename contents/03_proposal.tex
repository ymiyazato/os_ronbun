\section{提案} \label{section:proposal}
本研究ではヒュージページを有効的に活用するインメモリ分散処理フレームワークを提案する.
アプローチとしてはヒュージページの効果を受けやすいようなデータはヒュージページで生成して,
それ以外は通常のページで割り当てすることで効率の良いヒュージページ割り当てを行う.
提案方式によってヒュージページの恩恵も受けつつもデメリットを最小限に抑えることができる.

提案を実現する際にデザインチャレンジとなるのが以下のものである.

\begin{itemize}
  \item \emph{ヒュージページの割り当てはアプリケーション単位でしか選択できない}\\
  Apache Sparkでヒュージページを使う際には実行時にJavaオプションでUseTransParentHugepagesを指定して実行する.
  そうするとTHPによってJVM全体がヒュージページに割当たることになる.この選択はJVMごとにしか選べないため
  データ単位で割り当てることが出来ずSparkの特性に適したデータごとの割り当てを行うことができない.
  そこでデータ単位でヒュージページ割り当てを行う手法が必要となってくる.
  \item \emph{ヒュージページの効果を受けやすい適したデータをどう選ぶか}\\
  データの一部をヒュージページに割り当てることが出来たとして,どのデータに割り当てすれば
  効率的にヒュージページが使えるかを考える必要がある.
\end{itemize}



