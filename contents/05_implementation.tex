\section{実装} \label{section:implementation}
本研究では,Apache Spark3.2とOpenJDK11\cite{open-jdk}を実装対象とし,OpenJDKにデータ単位でのヒュージページ割り当てを
行うための機能を追加し,Sparkにその機能を用いて繰り返し利用されるデータに対してヒュージページを
割り当てるように変更を行った.

\subsection{JVMの改良}
Javaではオブジェクトを生成する際にnew命令を使って生成を行う.このnew命令に加えてオブジェクトを生成する際に
ヒュージページで生成するhp\_new命令を実装した.また,配列を生成するnewarray命令などもあるため,それぞれの
生成命令にはヒュージページで生成する命令を実装している.
実装のためにまず,JVMにHugepageRegionを追加した.Openjdk11の標準GCであるG1GC\cite{detlefs2004garbage}ではメモリがRegionという単位で
分割され管理されており,HugepageRegionはこのRegionをヒュージページとしたものである.
hp\_new命令が初めて呼び出されると通常のRegionを保存するフリーリストの中から使用したことのない
Regionを選びHugepageRegion専用のフリーリストに移動する.
一度別の用途に使われた後にフリーリストに返されたRegionだとヒュージページを割り当てる際に直接
ヒュージページ割り当てが出来ず,昇格によってヒュージページになるため優先して使用したことのないRegionが
選択される.
選択されたRegionはmadviseシステムコールによってMADV\_HUGEPAGEアドバイスが追加される.これによりこのRegionに
メモリを割り当てる際にはヒュージページとして割り当てられるようになる.
以降はhp\_new命令が呼び出された時はHugepageRegionから割り当て,存在しない場合は同じようにフリーリストから
もらってくるという手順となっている.
また,HugepageRegionを返す場合はヒュージページ専用のフリーリストに返すことでヒュージページと通常のページの
分離を行っている.
これにより通常はnew命令による通常の割り当て,ヒュージページを使いたい時はhp\_new命令を使うことで
データ単位でヒュージページの使用の有無を選べるようになる.
また,追加の工夫としてHugepageRegionは最初からOld領域にするようにしている.HugepageRegionに保存されるデータは長期間利用されるのが
前提のため,最初からOld領域にすることでYoung領域でGCを何度もされてOld領域へ移動といった手間を省くことができる.

\subsection{Sparkの改良}
Spark側ではStorageMemoryのデータを生成する際にhp\_new命令を使う用にする.
ここでhp\_new命令を使うケースとして二つ存在している.
まず,StorageMemoryに保存したいオブジェクトを自身のコード内で生成している場合である.
このような場合はユーザがhp\_new命令をコード内で記述してヒュージページとしての生成を行うことになる.
これによりhp\_new命令を使用したオブジェクトがヒュージページに割当たるようになる.
2つ目のケースがファイルなどに保存しておいたデータをシリアライズしてStorageMemoryにRDDとして保存する
場合である.このケースではSpark内でデータをキャッシュする際に利用するバッファーの生成に
hp\_newを使うようになっている.このバッファーがヒュージページで確保されることでデータがヒュージページとして
割当たることになる.
この場合ではユーザがコードの変更をすることなくStorageMemoryのデータをヒュージページとして割り当てることができる.



