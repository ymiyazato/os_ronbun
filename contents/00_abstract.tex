\begin{abstract}
  分散処理システムとは,複数のコンピュータで処理を分散して行うシステムで,大規模なデータを
  処理することができ,サーバのコスト削減やリソースの拡張性といったメリットが存在する.
  中でもApache Sparkのようなインメモリ分散処理フレームワークでは,処理を出来る限り
  メモリ内で行うことで,従来のものよりも高速な処理が可能となる.
  インメモリ分散処理では膨大なデータをメモリ上で扱う都合上,アドレス変換が大きなボトルネックになってしまう.
  アドレス変換のボトルネックを解消する機構としてヒュージページがある.
  Apache Sparkにヒュージページを適用する場合,主にTHPというヒュージページ管理機構が
  用いられるが,THPにはメモリ肥大化や,メモリのコンパクションのためにCPUが占有されるなど様々な問題が
  存在している.
  Apache SparkにおいてもTHPによるヒュージページ割り当ての効果は大きいが,THPのデメリットも受けてしまうため
  闇雲にヒュージページを利用することはできない.  
  本研究ではインメモリ分散処理フレームワーク上でヒュージページを効率的に利用できる手法を提案する.
  提案方式では,Apache Sparkのデータ領域の中でも,ヒュージページの効果が大きい領域にのみ
  ヒュージページの割り当てを行い,それ以外では通常のページ割り当てを行うことで無駄なヒュージページ割り当てを避ける.
  これにより,メモリ肥大化やコンパクションなどのデメリットを抑えつつ,ヒュージページによる性能向上の恩恵を受けることを狙う.
  実験の結果,  マイクロベンチマークのようなキャッシュデータへのアクセス頻度が高い
  ワークロードでは一部のみのヒュージページ割り当てで全体割り当てと同等以上のパフォーマンスが得られた.
\end{abstract}

%% キーワード (1--5単語) の記載は任意.
\begin{jkeyword}
分散処理システム,ヒュージページ
\end{jkeyword}

%\begin{eabstract}
%\end{eabstract}

%% the following keyword part is optional and can be omitted.

%\begin{ekeyword}
%  System security
%\LaTeX, style files
%\end{ekeyword}

%% if you use english opsion, you should put your English abstract in the abstract environment.
%% eabstract is not displayed in english mode
